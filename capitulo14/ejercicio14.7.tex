\input ../auxiliar

\titulo{Ejercicio 14.7}

\bigskip

\enunciadoS Explique el uso de |\quad| en este poema:

|Roses are red,|

|\quad Violets are blue;|

|Rhymes can typeset|

|\quad With boxes and glue.|

\noindent?`Qu\'e hubiera pasado si `|\quad|' hubiera sido reemplazado
por `|\indent|' en ambos sitios?

\bigskip

\respuesta{}

{\obeylines\smallskip
Roses are red,
\quad Violets are blue;
Rhymes can typeset
\quad With boxes and glue.
\smallskip}
\medskip

\hrule

{\obeylines\smallskip
Roses are red,
\indent Violets are blue;
Rhymes can typeset
\indent With boxes and glue.
\smallskip}
\medskip

\hrule

\medskip

Los poemas anteriores muestran el uso del comando |\obeylines| y como
cambia la identaci\'on del texto. En el primer poema, en la segunda y
cuarta l\'{\i}nea se muestra el uso del comando |\quad| que agrega un
espacio en blanco de 1em (para verificar esto, usar el comando {\sl
tex} y luego escribir en la c\'onsola interactiva: |\show\quad|).

{\bf Pendiente (TO-DO):} ?`Por qu\'e |\indent| en el segundo poema no
agrega espacio en la segunda y cuarta l\'{\i}nea (|\indent| funciona
solo en la primera l\'{\i}nea de los p\'arrafos, pero hice la prueba
aplic\'andolo a la primera l\'{\i}nea tambi\'en y no pas\'o nada)?

\bye

