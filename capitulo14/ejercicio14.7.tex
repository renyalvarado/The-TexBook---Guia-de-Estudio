{\obeylines\smallskip
Roses are red,
\quad Violets are blue;
Rhymes can typeset
\quad With boxes and glue.
\smallskip}
\medskip

\hrule

{\obeylines\smallskip
Roses are red,
\indent Violets are blue;
Rhymes can typeset
\indent With boxes and glue.
\smallskip}
\medskip

\hrule

\medskip

Los poemas anteriores muestran el uso del comando {\tt \string obeylines} y como cambia
la identaci\'on del texto. En el primera poema, en la segunda y cuarta l\'{\i}nea se 
muestra el uso del comando {\tt \string\quad} que agrega un espacio en blanco de 1em (para
verificar esto, usar el comando {\sl tex} y luego escribir en la c\'onsola interactiva:
{\tt \string\show\string\quad}).

{\bf Pendiente (TO-DO):} ?`Por qu\'e {\tt \string\indent} en el segundo poema no agrega espacio
en la segunda y cuarta l\'{\i}nea ({\tt \string\indent} funciona solo en la primera l\'{\i}nea
de los p\'arrafos, pero hice la prueba aplic\'andolo a la primera l\'{\i}nea tambi\'en y no pas\'o
nada)?
\bye