\input ../auxiliar
\titulo{Ejercicio 8.2}

\bigskip

\enunciadoS Prueba tu entendimiento de las reglas de lectura de {\TeX} respondiendo las siguientes preguntas:

\noindent a) ?`Cu\'al es la diferencia entre las categor\'{\i}as 5 y
14?

\noindent b) ?`Cu\'al es la diferencia entre las categor\'{\i}as 3 y
4?

\noindent c) ?`Cu\'al es la diferencia entre las categor\'{\i}as 11 y
12?

\noindent d) ?`Son los espacios en blanco ignorado despu\'es de
caracteres activos ?

\noindent e) ?`Cuando una l\'{\i}nea finaliza con caracter comentario
como |%|, son los espacios en blanco ignorados al comienzo de la
siguiente l\'{\i}nea?

\noindent f) ?`Puede un caracter ignorado aparacer en el medio de un
nombre de una secuencia de control?

\bigskip

\respuestaS
\noindent a) La categoría 5 termina la línea y es sustiuido luego por
un caracter en blanco (categor%ía 10), sin embargo el comentario
aunque igualmente termina la línea no sustituido por ningún otro
caracter, tan solo es ignorado.

\noindent Ejemplo:

\noindent a.1) Con caracter con categor\'{\i}a 5 como finalizador de l\'{\i}nea: se muestra el texto de las dos l\'{\i}neas con un espacio en blanco entre ellas:

\noindent |Linea 1|

\noindent |Linea 2|

Linea 1
Linea 2

\medskip

\noindent a.2) Con caracter con categor\'{\i}a 10 como finalizador de l\'{\i}nea, se muestra el texto de las dos l\'{\i}neas junto:

\noindent |Linea 1%|

\noindent |Linea 2|

\medskip

\noindent b) TO-DO

\medskip

\noindent c) TO-DO

\medskip

\noindent d) TO-DO

\medskip

\noindent e) TO-DO

\medskip

\noindent f) TO-DO

\bye

