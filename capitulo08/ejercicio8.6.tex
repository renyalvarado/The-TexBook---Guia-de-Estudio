\input ../auxiliar
\titulo{Ejercicio 8.5}

\bigskip

\enunciadoS Asuma que los c\'odigos de categor\'{\i}a de plain {\TeX}
est\'an em vigor, excepto que los caracteres |^^A|, |^^B|, |^^C| y
|^^M| pertenecen respectivamente a las categor\'{\i}as 0, 7, 10 y
11. ?`Qu\'e tokens son producidos desde (la bastante rid\'{\i}cula)
l\'{\i}nea de entrada`|^^B^^BM^^A^^B^^C^^M^^@\M|\literalblanco'?
(Recuerde que esta l\'{\i}nea es seguida por $\langle$return$\rangle$,
que es |^^M| y recuerde que |^^@| denota el caracter
$\langle$null$\rangle$, que tiene categor\'{\i}a 9 cuando |INITEX|
inicia.)

\bigskip

\respuestaS

\char`^\char`^B$_{_7}$ % Se cambió la categoría a 7
\char`^\char`^B$_{_7}$ % Ídem
M$_{_{11}}$ % Caracter 
\char`^\char`^B % Secuencia de Control, ya que ^^A ahora tiene código 0 y ^^C hace que finalice por tener código 10
\literalblanco$_{_{10}}$ % Blanco por tener ^^C código 10
M$_{_{11}}$ % Caracter, luego viene ^^@ que se ignora por pertenecer a la categoría 9
M\char`^\char`^M % Secuencia de control, se agrega ^^M al final de la línea y se elimina el espacio en blanco por reglas de TeX.

\bye

