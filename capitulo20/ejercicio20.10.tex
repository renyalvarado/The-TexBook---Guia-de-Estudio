\input ../auxiliar

\titulo{Ejercicio 20.10}

\bigskip

\enunciadoS Contin\'ue el ejemplo de IRS, asuma que |\fullypaid| y
|\underpaid| est\'an definidos como:
\medskip
|\def\fullypaid{Your taxes are fully paid---thank you.}|
\medskip
|\def\underpaid{{\count0=-balance|

\quad|\ifnum\count0<100|

\qquad|You owe \dollaramount, but you need not pay it, because|

\qquad|our policy is to disregard amounts less than \$1.00.|

\quad|\else Please remit \dollaramount within ten days,|

\qquad|or additional interest charges will be due.\fi}}|
\medskip

\noindent Escriba una macro |\overpaid| y asuma que |\dollaramount| es
una macro que genera el contenido de |\count0| en d\'olares y
centavos. La macro deber\'{\i}a decir que un cheque ser\'a enviado por
correo en un sobre aparte al menos que el monto sea menor que \$1.00,
en cuyo caso la persona debe solicitar expl\'{\i}citamente un cheque.

\bigskip

\countdef\balance=100
\balance=1000

\def\overpaid{{\count0=\balance
  \ifnum\count0<100
    Usted ha pagado \dollaramount de m\'as. Si desea un cheque por ese 
    monto, entonces realice una solicitud, ya que no se acostumbra 
    enviar cheques de manera autom\'atica cuando el monto es inferior
    a 1\$.00.
  \else Usted ha pagado \dollaramount\/ de m\'as. Un cheque con ese 
    monto ser\'a enviado autom\'aticamente .\fi}}
%\overpaid

\respuestaS


|\def\overpaid{{\count0=\balance|

\quad|\ifnum\count0<100|

\qquad|Usted ha pagado \dollaramount de m\'as. Si desea un cheque por ese|

\qquad|monto, entonces realice una solicitud, ya que no se acostumbra|

\qquad|enviar cheques de manera autom\'atica cuando el monto es inferior|

\qquad|a 1\$.00.|

\quad|\else Usted ha pagado \dollaramount\/ de m\'as. Un cheque con ese|

\qquad|monto ser\'a enviado autom\'aticamente .\fi}}|

\bye

