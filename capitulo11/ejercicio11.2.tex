\input ../auxiliar

\titulo{Ejercicio 11.2}

\bigskip

\enunciado{Mediante la ejecuci\'on de {\TeX}, averig\"ue c\'omo se
maneje la correciones it\'alicas a caracteres: ?`C\'omo son las
correciones representadas dentro de un box?}

\bigskip

\respuestaS Para realizar la prueba utilizamos una combinaci\'on de
dos comandos como: |\showbox| que nos permite ver los boxes y
|\batchmode| para que entre en modo lote y no se detenga al utilizar
|\showbox| sino que redirija todos los mensajes al archivo de log. El
espacio depende de la letra, por ejemplo al usar la letra inclinada
`|a|', no se agrega espacio al final(|.\kern 0.0|), sin embargo si la
letra es `g', entonces el espacio blanco es más grande: |.\kern
0.85649|.

\medskip

Ejemplo utilizado

|\batchmode |

|\setbox 0=\hbox{\sl g\/} \showbox 0|

\batchmode % 
\setbox 0=\hbox{\sl g\/} \showbox 0

\bye

