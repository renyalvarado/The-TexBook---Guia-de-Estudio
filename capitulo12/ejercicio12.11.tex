\def\nombre#1{{\it #1\/}}
\def\comando#1{comando {\tt \string#1\relax}}
Las dimensiones de \nombre{box3} son: 2pt de profundidad, 3pt de alto y 4 de ancho. Los 2 puntos de profundidad vienen 
dado por la profundidad de la \'ultima caja(\nombre{box2}) y se toma ese valor ya que no viene seguida de kerning ni de 
glue (regla definida en la página 80). Los 3ptos de alto resultan de la definici\'on  de altura de la  caja \nombre{box3}
y finalmente los 4 ptos de ancho se originan del desplazamiento en en 3ptos de la caja \nombre{box1} mas sun ancho (1 pto), 
$ 3 + 1 = 4$ ptos.
\end
