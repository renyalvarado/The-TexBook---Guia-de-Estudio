\input ../auxiliar

\titulo{Ejercicio 12.11}

\bigskip

\enunciadoS Asuman que |\box1| tiene 1 pt de altura, 1 pt de
profundidad y 1 pt de ancho; baselineskip, lineskip y lineskiplimit
son todas cero y |\boxmaxdepth| es muy grande. Una tercera caja ({\it
box}) es definida como:

\medskip 

|\setbox3=\vbox to3pt{\moveright3pt\box1\vskip-3pt plus3fil\box2}|

\medskip 

\noindent ?`Cu\'al es la altura, profundidad y anchura de |\box3|?
Describa la posici\'on de los puntos de referencia de las cajas ({\it
boxes}) 1 y 2 con respecto al punto de referencia de la caja 3.

\bigskip

\respuestaS Las dimensiones de |\box3| son: 2pt de profundidad, 3pt de
alto y 4 de ancho. Los 2 puntos de profundidad vienen dado por la
profundidad de la \'ultima caja(|\box2|) y se toma ese valor ya que no
viene seguida de kerning ni de glue (regla definida en la página
80). Los 3ptos de alto resultan de la definici\'on de altura de la
caja |\box3| y finalmente los 4 ptos de ancho se originan del
desplazamiento en en 3ptos de la caja |\box1| mas sun ancho (1 pto), $
3 + 1 = 4$ ptos.

\bye

