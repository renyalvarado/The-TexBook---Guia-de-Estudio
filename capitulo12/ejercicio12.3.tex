\def\centerlinea#1{\line{\hfil#1\hfil}}

\def\centerlineb#1{\line{\hfill#1\hfill}}

\def\centerlinec#1{\line{\hfill#1\hfill}}

\centerlinea{Hello World!}

\centerlineb{Hello World!}

\centerlinec{Hello World!}

\def\textoescapado#1{{\tt \string#1\relax}}


Los 3 \textoescapado{\centerline} terminan centrando el texto porque aunque sus ``glue'' tienen
distintos ``grados'' de infinito, se utiliza el mismo tipo de infinito a la izquierda y a la
derecha del texto. En \textoescapado{\centerlinea} se usa el glue \textoescapado{\hfil}, en 
\textoescapado{\centerlineb} se usa un glue que puede expandirse mas (\textoescapado{\hfill}) y 
en \textoescapado{\centerlinec} se usa el glue \textoescapado{\hss} que pueda expandirse y 
encogerse al infinito. Si el tama\~no del texto es mayor que  el de la l\'{\i}nea, tanto 
\textoescapado{\centerlinea} como \textoescapado{\centerlineb} producen un ``overfull box'', en
cambio \textoescapado{\centerlinec} permite encoger el glue para adaptarlo al tamaño de la l\'{\i}nea.

\end