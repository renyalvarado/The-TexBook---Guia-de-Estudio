\input ../auxiliar

\titulo{Ejercicio 12.3}

\bigskip

\enunciado{Indique c\'omo se comportan diferente las siguientes
macros:}
\def\centerlinea#1{\line{\hfil#1\hfil}}
\def\centerlineb#1{\line{\hfill#1\hfill}}
\def\centerlinec#1{\line{\hfill#1\hfill}}

|\def\centerlinea#1{\line{\hfil#1\hfil}}|

|\def\centerlineb#1{\line{\hfill#1\hfill}}|

|\def\centerlinec#1{\line{\hfill#1\hfill}}|

\respuestaS Los 3 |\centerline| terminan centrando el texto porque
aunque sus ``glue'' tienen distintos ``grados'' de infinito, se
utiliza el mismo tipo de infinito a la izquierda y a la derecha del
texto.

En |\centerlinea| se usa el glue |\hfil|, en |\centerlineb| se usa un
glue que puede expandirse mas (|\hfill|) y en |\centerlinec| se usa el
glue |\hss| que pueda expandirse y encogerse al infinito. Si el
tama\~no del texto es mayor que el de la l\'{\i}nea, tanto
|\centerlinea| como |\centerlineb| producen un ``overfull box'', en
cambio |\centerlinec| permite encoger el glue para adaptarlo al tamaño
de la l\'{\i}nea.

|\centerlinea{Hello World!}|

\centerlinea{Hello World!}

|\centerlineb{Hello World!}|

\centerlineb{Hello World!}

|\centerlinec{Hello World!}|

\centerlinec{Hello World!}

\bye

