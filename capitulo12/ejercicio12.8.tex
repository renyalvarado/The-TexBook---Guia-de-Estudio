\input ../auxiliar

\titulo{Ejercicio 12.8}

\bigskip

\enunciadoS Asuman que |\box1| tiene 1 pt de altura, 1 pt de
profundidad y 1 pt de ancho; |\box2| tiene 2 pt de altura, 2 pt de
profundidad y 2 pt de ancho. Una tercera caja ({\it box}) es creada de
la siguiente manera:

\medskip

|\setbox3=\hbox to3pt{\hfil\lower3pt\box1\hskip-3pt plus3fil\box2}|

\medskip

\noindent ?`Cu\'al es la altura, profundidad y anchura de |\box3|?
Describa la posici\'on de los puntos de referencia de las cajas ({\it
boxes}) 1 y 2 con respecto al punto de referencia de la caja 3.

\bigskip

\respuestaS Las dimensiones de |\box3| son: 4 de profundidad, 2pt de
alto y 3 de ancho. Los 4 cuatro de profundidad viene dado porque el
|\lower3pt| ``baja'' la l\'{\i}nea base de |\box1| 3 puntos, y como
esta caja tiene 1pto de profundidad por s\'{\i} misma, tenemos $3 + 1
= 4$ puntos de profundidad. Los 2 puntos de altura equivalen a los de
|\box2| (esta caja mantiene su posici\'on original porque el |\lower|
solo afecta al siguiente argumento, en este caso a
|\box1|). Finalmente los 3 pt de ancho es por el |\hbox{} to 3pt| que
establece el tama\~no de la caja contenedora.

\bye

