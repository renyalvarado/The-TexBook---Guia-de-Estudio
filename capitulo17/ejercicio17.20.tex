\input auxiliar

\titulo{Ejercicio 17.20}

\bigskip

\enunciado{Comandos como \comando{\mathchardef}\comando{\alpha}{\tt=''010B} \char026 son usados en 
el Ap\'endice B para definir las letras griegas min\'usculas. Suponga que
quiere extender plain \TeX colocando las letras metem\'aticas it\'alicas 
negrillas en la familia 9, analogamente a las letras matem\'aticas it\'alicas
en la familia 1 (estas fuentes no est\'an disponbibles en las versiones 
simplificadas de \TeX, pero asumamos que existen). Asuma que la secuencia de
control \comando{\bmit} ha sido definida como una abreviatura de `{\tt 
\comando{\fam}=9}', por lo tanto `{\tt\char`\{\comando{\bmit} b\char`\}}'
generar\'a una  {\tt b } matem\'atica it\'alica negrilla. ?`Qu\'e cambio a la 
definici\'on de \comando{\alpha} har\'a que 
{\tt\char`\{\comando{\bmit}\comando{\alpha}\char`\}} produzca un alfa en
negrilla.}

\bigskip

\respuesta{\comando{\mathchardef}\comando{\alpha}{\tt=''710B}.}

\bye
