\input ../auxiliar

\titulo{Ejercicio 17.20}

\bigskip

\enunciadoS Comandos como |\mathchardef\alpha=''010B| son usados en 
el Ap\'endice B para definir las letras griegas min\'usculas. Suponga que
quiere extender plain \TeX colocando las letras metem\'aticas it\'alicas 
negrillas en la familia 9, analogamente a las letras matem\'aticas it\'alicas
en la familia 1 (estas fuentes no est\'an disponbibles en las versiones 
simplificadas de \TeX, pero asumamos que existen). Asuma que la secuencia de
control |\bmit| ha sido definida como una abreviatura de `|\fam=9|', por lo tanto `|\bmit b}|'
generar\'a una  {\tt b } matem\'atica it\'alica negrilla. ?`Qu\'e cambio a la 
definici\'on de |\alpha| har\'a que |\bmit\alpha| produzca un alfa en
negrilla.

\bigskip

\respuestaS |\mathchardef\alpha=''710B|.

\bye

