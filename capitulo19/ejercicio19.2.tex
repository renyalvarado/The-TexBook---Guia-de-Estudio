\input ../auxiliar
\titulo{Ejercicio 19.2}

\bigskip

\enunciadoS Algunas veces el estilo `display' es muy pomposo cuando la
f\'ormula a ser mostrada es
$$y={1\over2}x$$
\noindent o algo igualmente simple. Un d\'{\i}a B.~L.~User intentó
remediar esto escribiendolo como `|$$y={\scriptstyle1|
|\over\scriptstyle2}x$$|', pero la f\'ormula
$$y={\scriptstyle1\over\scriptstyle2}x$$
\noindent no era lo que el ten\'{\i}a en mente.?`Cu\'al es la la
manera correcta de obtener simplemente `$y={1\over2}x$' cuando no
quieres fracciones grandes en modo `display'.

\bigskip

\respuestaS

|$$\textstyle y={1\over2}x$$|

$$\textstyle y={1\over2}x$$

\bye

