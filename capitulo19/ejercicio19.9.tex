\input ../auxiliar
\titulo{Ejercicio 19.9}

\enunciadoS En la pr\'actica, el lado izquierdo de f\'ormulas se
encuentra a menudo en blanco y el alineamiento es hecho en relaci\'on
a otros s\'{\i}mbolos distintos a |=|.  Por ejemplo, la siguiente
f\'ormula es t\'{\i}pica. Vea si puede acertar c\'omo el autor la
escribi\'o.
$$\eqalign{T(n)\le T(2^{\lceil\lg n\rceil})&\le
c(3^{\lceil\lg n\rceil} - 2^{\lceil\lg n\rceil})\cr
&<3c\cdot3^{\ln n}\cr
&=3c\,n^{\ln 3}.\cr}$$

\bigskip

\respuestaS

|$$\eqalign{T(n)\le T(2^{\lceil\lg n\rceil})&\le|

|c(3^{\lceil\lg n\rceil} - 2^{\lceil\lg n\rceil})\cr|

|&<3c\cdot3^{\ln n}\cr|

|&=3c\,n^{\ln 3}.\cr}$$|

\bye

