\input ../auxiliar

\titulo{Ejercicio 16.13}

\bigskip

\def\ghat{\hat g}
\enunciadoS Este ha sido otro cap\'{\i}tulo largo; pero
al\'egrate,!`has aprendido bastante! Prueba esto explicando c\'omo
escribir las siguientes f\'ormulas $e^{-x^2}$, $D \sim p^{\alpha} + M
+ l$, y $\ghat \in (H^{\pi_1^{-1}})'$. (En este ejemplo, asuma que ya
se ha definido la secuencia de control |\ghat|, de modo que |\ghat|
produzca la letra acentuada $\ghat$.

\bigskip

\respuestaS |$e^{-x^2}$, $D \sim p^{\alpha} + M + l$| y 
|$\ghat \in (H^{\pi_1^{-1}})'$|.

\bye

