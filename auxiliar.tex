%% Secuencia de control para mostrar literalmente un texto.
%% Tomado del archivo de macros de 'The Texbook': manmac.tex

\chardef\other=12
\def\ttverbatim{\begingroup
  \catcode`\\=\other
  \catcode`\{=\other
  \catcode`\}=\other
  \catcode`\$=\other
  \catcode`\&=\other
  \catcode`\#=\other
  \catcode`\%=\other
  \catcode`\~=\other
  \catcode`\_=\other
  \catcode`\^=\other
  \obeyspaces \obeylines \tt}

\outer\def\begintt{$$\let\par=\endgraf \ttverbatim \parskip=\z@
  \catcode`\|=0 \rightskip-5pc \ttfinish}
{\catcode`\|=0 |catcode`|\=\other % | is temporary escape character
  |obeylines % end of line is active
  |gdef|ttfinish#1^^M#2\endtt{#1|vbox{#2}|endgroup$$}}

\catcode`\|=\active
{\obeylines \gdef|{\ttverbatim \let^^M=\  \let|=\endgroup}}

\def\titulo#1{\centerline{\bf#1}}

\def\bloque#1#2{\noindent{\bf #1:} #2\par}

\def\enunciado#1{\bloque{Pregunta}{#1}}

\def\enunciadoS{\noindent{\bf Pregunta:} }

\def\respuesta#1{\bloque{Respuesta}{#1}}

\def\respuestaS{\noindent{\bf Respuesta:} }
