\input ../auxiliar
\titulo{Ejercicio 7.5}

\bigskip

\enunciadoS Experimente con {\TeX} para ver que hace |\string| cuando
es seguido por aun caracter activo como |~|. Los caracteres activos se
comportan como secuencias de control, pero ellos no llevan escape como
prefijo. ?`Cu\'al es una forma f\'acil de llevar a cabo un experimento
interactivo? Qu\'e secuencia de control se podr\'{\i}a poner despu\'es
de |\string| para obtener el token del caracter sencillo |\|$_{12}$.

\bigskip

\respuestaS Se puede usar |\message| para obtener valores de forma
interactiva:

\medskip

\noindent|\message{\string~} % Produce ~ en el archivo de log|
\message{\string~} % Produce ~ en el archivo de log

\medskip

\noindent|{ % Creamos un bloque para que los cambios realizados solo se apliquen allí|

\noindent|\catcode`*=0 % Ponemos otro caracter para indicar el inicio de una secuencia de control|

\noindent|\catcode`\\=13 % Colocamos al caracter / como elemento activo|

\noindent|*message{*string/} % Mostramos el caracter / en el log.|

\noindent|}|




{ % Creamos un bloque para que los cambios realizados solo se apliquen allí.
 \catcode`*=0 % Ponemos otro caracter para indicar el inicio de una secuencia de control, es decir le asignamos al grupo 0.
 \catcode`\\=13 % Colocamos al caracter / como elemento activo.
 *message{*string/} % Mostramos el caracter / en el log.
 }
\bye

