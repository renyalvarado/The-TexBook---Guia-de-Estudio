\input auxiliar

\titulo{Ejercicio 15.6}

\bigskip

\enunciado{Pruebe sus conocimientos de los registros de \TeX\/ indicando los resultados de cada uno de los siguientes comandos cuando son ejecutados en secuencia:}
{\tt \comandoigual{\count1}{50} \kern0.4cm \comandoigual{\dimen2}{\comando{\count1}pt} \kern0.4cm \comando{\divide}\comando{\count1} by 8

\comandoigual{\skip2}{-10pt plus \comando{\count1fil} minus \comando{\dimen2}}

\comando{\multiply}\comando{\skip2} by -\comando{\count1} \comando{\divide}\comando{\skip2} by \comando{\dimen2} \kern0.4cm \comando{\count6}=\comando{\skip2}

\comandoigual{\skip1}{.5\comando{\dimen2} plus\comando{\skip2} minus\comando{\count}\comando{\count1fill}}

\comando{\multiply}\comando{\skip2} by\comando{\skip1} \kern0.4cm \comando{\advance}\comando{\skip1} by-\comando{\skip2}}

\bigskip

\respuesta{Los resultados de los c\'alculos son: }
{\tt \comandoigual{\count1}{50}} \kern0.4cm  \count1=50   \the\count1

{\tt \comandoigual{\dimen2}{\comando{\count1}pt}} \kern0.4cm \dimen2=\count1pt  \the\dimen2

{\tt \comando{\divide}\comando{\count1} by 8} \kern0.4cm \divide\count1 by 8 \the\count1

\medskip

{\tt \comandoigual{\skip2}{-10pt plus \comando{\count1fil} minus \comando{\dimen2}}} \kern0.4cm \skip2=-10pt plus \count1fil minus \dimen2  \the\skip2

{\tt \comando{\multiply}\comando{\skip2} by -\comando{\count1}} \kern0.4cm \multiply\skip2 by -\count1 \the\skip2

{\tt \comando{\divide}\comando{\skip2} by \comando{\dimen2}} \kern0.4cm \divide\skip2 by \dimen2 \the\skip2sp

{\tt \comando{\count6}=\comando{\skip2}} \kern0.4cm \count6=\skip2 \the\count6

{\tt \comandoigual{\skip1}{.5\comando{\dimen2} plus\comando{\skip2} minus\comando{\count}\comando{\count1fill}}} \kern0.4cm \skip1=.5\dimen2 plus\skip2  minus \count\count1fill  \the\skip1

{\tt \comando{\multiply}\comando{\skip2} by\comando{\skip1}} \kern0.4cm \multiply\skip2 by \skip1 \the\skip2

{\tt \comando{\advance}\comando{\skip1} by-\comando{\skip2}} \kern0.4cm \advance\skip1 by \skip2 \the\skip1

\bye