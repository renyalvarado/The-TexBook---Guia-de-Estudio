\input ../auxiliar

\titulo{Ejercicio 3.4}

\bigskip

\enunciado{?`Cu\'antas secuencias de control diferentes de longitud 2
(incluyendo el caracter de escape) son posibles? ?`Cu\'antas de
longitud 3?}

\bigskip

\respuesta{Como tenemos que incluir el caracter de escape, tenemos que
s\'olo nos resta un solo caracter para que forme parte de la secuencia
de control. Con eso tendríámos 256 secuencias de control
(correspondiente al número de caracteres ASCII) distintas de las
cuales 52 corresponderían a palabras de control y el resto a símbolos
de control.}

Cuando es de longitud tres, la secuencia de control solo puede incluir
letras min\'usculas y may\'usculas (56 entre las dos) en los dos
\'ultimos caracteres. Por lo que el n\'umero total de secuencias es
52~$\times$52~$=$~2704.

\bye

